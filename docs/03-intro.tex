\section*{\large ВВЕДЕНИЕ}
\addcontentsline{toc}{section}{ВВЕДЕНИЕ}

По данным DatePortal, к январю 2022 года в Росссийской Федерации насчитывалось 129.8 миллионов интернет-пользователей, примерно 89\% популяции \cite{digital-2022}. При этом количество доменов в области российского интернета к декабрю 2022 года составляет 4.93 миллиона, при этом на большинстве этих сайтов (73.8\%) зарегистрированы физические лица и индивидуальные предприниматели \cite{forbes-ru}.  При этом объём трафика в рунете к 2022 году составил 91 эксабайт (квинтиллион байтов), а к концу 2023 года превысит 100 экабайт \cite{telecom-daily}.

В большинстве современных веб-приложений используются такие технологии, как HTML, JS и CSS, часто используются различные форматы файлов для отображения информации определённого типа (например, PNG, JPG или JPEG для изображений).

\textbf{Целью работы} является разработка сервера раздачи статческой информации на языке Си без использования сторонних библиотек, построенного при помощи паттерна \textit{thread pool} и использующего системный вызов \textit{pselect}. 
Для достижения поставленной цели необходимо выполнить следующие задачи:
\begin{enumerate}[label=\arabic*)]
	\item провести анализ предметной области;
	\item определить функционал, реализуемый сервером раздачи статической инфоормации;
	\item провести анализ паттерна \textit{thread pool} и системного вызова \textit{pselect} ;
	\item спроектировать и разработать сервер раздачи статическй информации;
	\item провести сравнение результатов нагрузочного тестирования при помощи Apache Benchmarks с NGINX.
\end{enumerate}

