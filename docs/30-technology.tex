\section{\large Технологическая часть}

В  данном  разделе  рассмотрены  средства  разработки  программы,  модули разработанного сервера и листинги исходных кодов.

Приведённые листинги исходных кодов включают в себя функции 

\subsection{Средства реализации}
Как  основное  средство  реализации  и  разработки  ПО  был  выбран  язык
программирования  Си.
Причиной  выбора  данного  языка  является следующие причины: во-первых, согласно требованиям к работе должен быть использован этот язык, во-вторых, язык Си компилируется напрямую в ассемблер, команды которого напрямую соответствуют машинным командам, что делает язык Си предпочтительным для написания сервера, поскольку это обеспечивает наименьшее время обработки запроса.

\subsection{Модули программы}

Реализованная программа состоит из 5 следующих модулей:
\begin{enumerate}[label=\arabic*)]
	\item main, содержащий точку входа в программу, а также основной цикл обработки сервера;
	\item server, содержащий набор функций для обработки запросов;
	\item thread\_pool, содержащий функцию создания пула потоков ;
	\item logger, содержащий функции для логирования запросов;
	\item http, содержащий необходимые для обработки HTTP запросов константы.
\end{enumerate}

\subsection{Реализация сервера}

В  расположенных  ниже  листингах \ref{code:1-1} -- \ref{code:1-4} приведены реализации основного цикла сервера, а в листинге \ref{code:2-1} --- 

\begin{figure}[H]
\begin{code}
	\captionsetup{justification=centering}
	\captionof{listing}{Реализация основного цикла сервера (часть 1	)}
	\label{code:1-1}
	\inputminted
	[
	frame=single,
	framerule=0.5pt,
	framesep=20pt,
	fontsize=\small,
	tabsize=4,
	linenos,
	numbersep=5pt,
	xleftmargin=10pt,
	]
	{text}
	{code/main-1.c}
\end{code}
\end{figure}

\begin{figure}[H]
\begin{code}
	\captionsetup{justification=centering}
	\captionof{listing}{Реализация основного цикла сервера (часть 2)}
	\label{code:1-2}
	\inputminted
	[
	frame=single,
	framerule=0.5pt,
	framesep=20pt,
	fontsize=\small,
	tabsize=4,
	linenos,
	numbersep=5pt,
	xleftmargin=10pt,
	]
	{text}
	{code/main-2.c}
\end{code}
\end{figure}

\begin{figure}[H]
\begin{code}
	\captionsetup{justification=centering}
	\captionof{listing}{Реализация основного цикла сервера (часть 3)}
	\label{code:1-3}
	\inputminted
	[
	frame=single,
	framerule=0.5pt,
	framesep=20pt,
	fontsize=\small,
	tabsize=4,
	linenos,
	numbersep=5pt,
	xleftmargin=10pt,
	]
	{text}
	{code/main-3.c}
\end{code}
\end{figure}

\begin{figure}[H]
\begin{code}
	\captionsetup{justification=centering}
	\captionof{listing}{Реализация основного цикла сервера (часть 4)}
	\label{code:1-4}
	\inputminted
	[
	frame=single,
	framerule=0.5pt,
	framesep=20pt,
	fontsize=\small,
	tabsize=4,
	linenos,
	numbersep=5pt,
	xleftmargin=10pt,
	]
	{text}
	{code/main-4.c}
\end{code}
\end{figure}

\begin{figure}[H]
\begin{code}
	\captionsetup{justification=centering}
	\captionof{listing}{Реализация функции обработки запроса в пуле потоков (часть 1)}
	\label{code:2-1}
	\inputminted
	[
	frame=single,
	framerule=0.5pt,
	framesep=20pt,
	fontsize=\small,
	tabsize=4,
	linenos,
	numbersep=5pt,
	xleftmargin=10pt,
	]
	{text}
	{code/thread-1.c}
\end{code}
\end{figure}

\begin{figure}[H]
\begin{code}
	\captionsetup{justification=centering}
	\captionof{listing}{Реализация функции обработки запроса в пуле потоков (часть 2)}
	\label{code:2-2}
	\inputminted
	[
	frame=single,
	framerule=0.5pt,
	framesep=20pt,
	fontsize=\small,
	tabsize=4,
	linenos,
	numbersep=5pt,
	xleftmargin=10pt,
	]
	{text}
	{code/thread-2.c}
\end{code}
\end{figure}

\begin{figure}[H]
\begin{code}
	\captionsetup{justification=centering}
	\captionof{listing}{Реализация функции обработки запроса в пуле потоков (часть 3)}
	\label{code:2-3}
	\inputminted
	[
	frame=single,
	framerule=0.5pt,
	framesep=20pt,
	fontsize=\small,
	tabsize=4,
	linenos,
	numbersep=5pt,
	xleftmargin=10pt,
	]
	{text}
	{code/thread-3.c}
\end{code}
\end{figure}

\begin{figure}[H]
\begin{code}
	\captionsetup{justification=centering}
	\captionof{listing}{Реализация функции обработки запроса в пуле потоков (часть 4)}
	\label{code:2-4}
	\inputminted
	[
	frame=single,
	framerule=0.5pt,
	framesep=20pt,
	fontsize=\small,
	tabsize=4,
	linenos,
	numbersep=5pt,
	xleftmargin=10pt,
	]
	{text}
	{code/thread-4.c}
\end{code}
\end{figure}



\subsection*{Вывод}
В данном разделе были рассмотрены  средства  разработки  программы,  модули разработанного сервера и листинги исходных кодов.
Разработанную программу следует протестировать при помощи нагрузочного тестирования с использованием Apache Benchmarks, а также сравнить результаты тестирования с тестированием сервера раздачи статической информации, реализованного при помощи NGINX.

\pagebreak